\documentclass[../main.tex]{subfiles}
\graphicspath{{\subfix{}}}

\begin{document}
\setbool{@fleqn}{false} %THIS DETERMINES IF EQUATIONS ARE CENTERED!
%\chapter{Current Chapter Name} %If this is the first section in the document, it needs a chapter header, activate this to see how the spacing will be effected, but make sure to disable it before rendering format.tex

\section{Envelope Detector Examples}

%----------FIGURE EXAMPLES----------

%Use this code for a figure displaying an image in the Images folder

%\begin{figure}[H]
%\centering
%\includegraphics[scale=0.5]{../Images/NAME}
%\caption{FIGURE CAPTION}
%\label{FIGURE NAME (not visible)}
%\end{figure}

%Use this code for a figure displaying a figure in the Figures folder

%\begin{figure}[H]
%\centering
%\scalebox{1.0}{\subfile{../Figures/NAME}}
%\caption{FIGURE CAPTION}
%\label{FIGURE NAME (not visible)}
%\end{figure}

%----------START OF SECTION----------

\noindent Note: All figures for this section are found in Figures/Envelope$\_$Detector

\vspace{1\baselineskip}

\noindent\textbf{Realistic Circuit:} Below is a simple ``Envelope Detector" circuit, Figure ``Envelope$\_$Circuit$\_1$". This circuit was simulated in PSpice, resulting in the plots below. These plots import from the tables in the ``Data" folder for Circuit 1.

\begin{figure}[H]
\centering
\scalebox{1.0}{\subfile{../Figures/Envelope_Detector/Envelope_Circuit_1}}
\caption{Ideal Envelope Detector Circuit}
\label{Envelope_Circuit_1-Workshop}
\end{figure}

\begin{figure}[H]
\centering
\scalebox{0.5}{\subfile{../Figures/Envelope_Detector/y2_Circuit_1}}
\caption{Ideal Circuit: $y_1(t)$ vs $y_2(t)$ with $V_F=0$V}
\label{y2_Circuit_1-Workshop}
\end{figure}

\begin{figure}[H]
\centering
\scalebox{0.5}{\subfile{../Figures/Envelope_Detector/y2_Circuit_1_diff}}
\caption{Ideal Circuit: Loss from Diode}
\label{y2_Circuit_1_diff-Workshop}
\end{figure}

\begin{figure}[H]
\centering
\scalebox{0.5}{\subfile{../Figures/Envelope_Detector/y2_Circuit_1_loss}}
\caption{Ideal Circuit: $y_1(t)$ vs $y_2(t)$ with with $V_F=0.26$V}
\label{y2_Circuit_1_loss-Workshop}
\end{figure}

\newpage
\noindent\textbf{Realistic Circuit:} A more practical circuit is shown below in Figure \ref{Envelope_Circuit_2-Workshop}, and the same circuit with a `UA741 Op-Amp Adder Circuit' added to the left is shown in Figure \ref{Envelope_Circuit_3-Workshop}. The adder circuit creates and AM signal with message $m(t)$ and carrier $c(t)$, and the circuit shown in Figure \ref{Envelope_Circuit_2-Workshop} approximately recovers $m(t)$. Circuit 3 was simulated in PSpice, and similarly with Circuit 1, the resulting plots are shown below. These examples pull from the other data file, exported from PSpice.

\begin{figure}[H]
\centering
\scalebox{1.0}{\subfile{../Figures/Envelope_Detector/Envelope_Circuit_2}}
\caption{Practical Envelope Detector Circuit}
\label{Envelope_Circuit_2-Workshop}
\end{figure}

\begin{figure}[H]
\centering
\scalebox{1.0}{\subfile{../Figures/Envelope_Detector/Envelope_Circuit_3}}
\caption{Complete Envelope Detector Circuit}
\label{Envelope_Circuit_3-Workshop}
\end{figure}

\begin{figure}[H]
\centering
\scalebox{0.5}{\subfile{../Figures/Envelope_Detector/y1_Circuit_3}}
\caption{Envelope Detector: $y_0(t)$ vs $y_1(t)$}
\label{y1_Circuit_3-Workshop}
\end{figure}

\begin{figure}[H]
\centering
\scalebox{0.5}{\subfile{../Figures/Envelope_Detector/y2_Circuit_3}}
\caption{Envelope Detector: $y_1(t)$ vs $y_2(t)$}
\label{y2_Circuit_3-Workshop}
\end{figure}

\begin{figure}[H]
\centering
\scalebox{0.5}{\subfile{../Figures/Envelope_Detector/y3_Circuit_3}}
\caption{Envelope Detector: $y_2(t)$ vs $y_3(t)$}
\label{y3_Circuit_3-Workshop}
\end{figure}

\begin{figure}[H]
\centering
\scalebox{0.5}{\subfile{../Figures/Envelope_Detector/y4_Circuit_3}}
\caption{Envelope Detector: $y_3(t)$ vs $y_4(t)$}
\label{y4_Circuit_4-Workshop}
\end{figure}


\end{document}
