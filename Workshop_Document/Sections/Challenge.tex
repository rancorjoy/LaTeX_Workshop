\documentclass[../main.tex]{subfiles}
\graphicspath{{\subfix{}}}

\begin{document}
\setbool{@fleqn}{false} %THIS DETERMINES IF EQUATIONS ARE CENTERED!
%\chapter{Current Chapter Name} %If this is the first section in the document, it needs a chapter header, activate this to see how the spacing will be effected, but make sure to disable it before rendering format.tex

\section{Self Guided Challenges}
\noindent Using the example figures from the last three sections as a guide, feel free to attempt to replicate these figures. These (like the example figures) are all from my homework, and as such I know for a fact they are all doable using the same methods as before. For the last example you may need to consult the CircuitTikZ documentation.

\vspace{1\baselineskip}

\noindent\textbf{Challenge 1:} The provided image is an example of sampling a sinusoidal signal with a sampling period 1/T. The sampling function $p(t)$ has a duty cycle $d=0.25$ and a height $h=1$. Make the amplitude and frequency of the sinusoidal function parameterized so they can be changed if needed.

\begin{figure}[H]
\centering
\includegraphics[scale=0.65]{../Images/Challenge_1}
\caption{Challenge Prompt 1}
\label{challenge_image_1-Workshop}
\end{figure}

\begin{figure}[H]
\centering
\scalebox{0.5}{\subfile{../Figures/Challenge/Challenge_1}}
\caption{Replication of Prompt 1}
\label{challenge_figure_1-Workshop}
\end{figure}

\newpage
\noindent\textbf{Challenge 2:} The provided image is a circuit I took from one of my Communication Systems lab reports. It can take a sampled circuit like the one above, and render a DSB-SC AM signal from it. Replicate this schematic below.

\begin{figure}[H]
\centering
\includegraphics[scale=0.75]{../Images/Challenge_2}
\caption{Challenge Prompt}
\label{challenge_image_2-Workshop}
\end{figure}

\begin{figure}[H]
\centering
\scalebox{0.5}{\subfile{../Figures/Challenge/Challenge_2}}
\caption{Replication of Prompt 2}
\label{challenge_figure_2-Workshop}
\end{figure}

\newpage
\noindent\textbf{Challenge 3:} The provided image comes from a homework problem assigned in the elective class 1Intro to Photonics'. The provided image in the assignment was rather low resolution like the one given, so I replicated it in \LaTeX for my submission (because why not?). Replicate this figure below:

\begin{figure}[H]
\centering
\includegraphics[scale=0.75]{../Images/Challenge_3}
\caption{Challenge Prompt 3}
\label{challenge_image_3-Workshop}
\end{figure}

\begin{figure}[H]
\centering
\scalebox{0.5}{\subfile{../Figures/Challenge/Challenge_3}}
\caption{Replication of Prompt 3}
\label{challenge_figure_3-Workshop}
\end{figure}

\end{document}
