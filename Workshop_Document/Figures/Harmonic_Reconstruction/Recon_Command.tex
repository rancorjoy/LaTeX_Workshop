%Command to run the "Recon" figure script with parameter x-limits, harmonics and cuttoff frequency
%Entered as LPFrecon{x-limits}{harmonics}{cuttoff}
\newcommand{\LPFrecon}[3] %Shortcut for multicols setup
{

% Plot parameters
\pgfmathsetmacro\xbound{#1}   % x-axis limits
\pgfmathsetmacro\n{#2}     	  % number of harmonics in the reconstruction
\pgfmathsetmacro\wc{#3/(2*pi)} % cutoff in harmonic units k_c = wc/w0 (multiple of fundamental frequency)

% Define vi(t) coefficient for harmonic n
\pgfmathdeclarefunction{vinmag}{1}{%
  \pgfmathparse{1/(pi*abs(#1))}%
}

% Define vi(t) phase for harmonic n
\pgfmathdeclarefunction{posneg}{1}{%
  \pgfmathparse{#1 < 0 ? pi/2 : -pi/2}%
}
\pgfmathdeclarefunction{vinphase}{1}{%
  \pgfmathparse{mod(#1,2) == 0 ? 0 : posneg(#1)}%
}

% Period and base radian frequency
\pgfmathsetmacro\T{1}
\pgfmathsetmacro\wzero{2*pi/\T} % w0

% Define low pass filter magnitude for harmonic n
\pgfmathdeclarefunction{hmag}{1}{%
  \pgfmathparse{\wc/sqrt((#1)^2 + (\wc)^2)}%
}

% Define low pass filter phase for harmonic n
\pgfmathdeclarefunction{hphase}{1}{%
  \pgfmathparse{-rad(atan(#1/\wc))}%
}

% Define reconstruction of input signal
\def\vinrecon{0}
\foreach \m in {1,...,\n} {%
  \pgfmathparse{mod(\m,2)==1 ? (vinmag(\m)) : 0} \let\A\pgfmathresult
  \xdef\vinrecon{\vinrecon
    + (\A)*cos(deg(\m*\wzero*x - pi/2))
    + (\A)*cos(deg(-\m*\wzero*x + pi/2))}%
}

% Define reconstruction of output signal
\def\voutrecon{0}
\foreach \m in {1,...,\n} {%
  \pgfmathparse{mod(\m,2)==1 ? (vinmag(\m)) : 0} \let\A\pgfmathresult
  \xdef\voutrecon{\voutrecon
    + (\A*hmag(\m))*cos(deg(\m*\wzero*x - pi/2 + hphase(\m)))
    + (\A*hmag(\m))*cos(deg(-\m*\wzero*x + pi/2 - hphase(\m)))}%
}


\begin{tikzpicture}
\begin{axis}[
  width=0.95\linewidth,
  height=0.6\linewidth,
  samples=800,  
  xmin=-\xbound, xmax=\xbound,
  ymin=-1.1, ymax=1.1,
  axis x line=middle,
  axis y line=middle,
  xtick=\empty,
  xticklabels={},
  ytick={},
  yticklabels={},
  xlabel=$t$,
  ylabel=$v(t)$,
]

% Plot the partial sum f_n(x)
  \addplot[thick, domain=-\xbound:\xbound, gray] {\vinrecon};
  \addlegendentry{$v_i(t)$}
  \addplot[thick, domain=-\xbound:\xbound, black] {\voutrecon};
  \addlegendentry{$v_o(t)$}

\end{axis}
\end{tikzpicture}

}
