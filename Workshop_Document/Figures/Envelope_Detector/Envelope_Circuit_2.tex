\documentclass[../../main.tex]{subfiles}
\graphicspath{{\subfix{}}}

\begin{document}

\begin{circuitikz}[american,littleR/.style={R, resistors/scale=0.5}]

\ctikzset{capacitors/scale=0.5}
\ctikzset{
  sources/scale=0.85,              % a touch bigger so +/– aren’t cramped
  sources/symbol/thickness=0.8,    % optional: slimmer internal marks
  sources/symbol/sign rotation=straight % optional: always-horizontal +/–
}

% Positions (from op-amp circuit)
\draw (0, -2) coordinate (p1); % Ground Height
\draw (0, 0) coordinate (p3);  % Input

% Envelope Detector
\draw (p3) to [short, -o] (-0.1, 0) node(vin)[left]{$y_1(t)$};
\draw (p3) to [diode, l={1N5817}] ++ (4, 0) coordinate (p4);
\draw (p4) to [C, label={$C_1$}] ++ (0, -2) to [short] (p4 |- p1) node(p5)[ground]{};
\draw (p4) to [short] ++ (1.5, 0) to [R, l=$R_1$] ++ (0, -2) to [short] ++ (-1.5, 0);
\draw (p4) ++ (1.5, 0) to [C, l=$C_2$] ++ (2, 0) coordinate (y3) 
						to [R, l=$R_2$] ++ (0, -2) to [short] ++ (-2, 0);
\draw (y3) to [R, l=$R_3$] ++ (2.5, 0) coordinate (y4) 
						to [C, l=$C_3$] ++ (0, -2) to [short] ++ (-2.5, 0);

% Text
\node[above] at (p4){$y_2(t)$};
\node[above] at (y3){$y_3(t)$};
\node[above] at (y4){$y_4(t)$};


\end{circuitikz}

\end{document}